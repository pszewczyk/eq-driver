\documentclass[a4paper,11pt,twoside,openany]{book}
\usepackage[english,polish]{babel}
\usepackage[T1]{fontenc}
\usepackage[utf8]{inputenc}
\usepackage{gensymb}
\usepackage{graphicx}
\usepackage{svg}
\usepackage{wasysym}
\usepackage{listings}
\usepackage{tabularx}
\usepackage{fancyhdr}
\usepackage{titlesec}
\usepackage{geometry}
\graphicspath{ {images/} }

\assignpagestyle{\chapter}{fancy}

\lstset{language=C,
	tabsize=2,
	basicstyle=\footnotesize\ttfamily}

\geometry{
 a4paper,
 inner=30mm,
 outer=20mm,
 bottom=25mm,
 top=25mm,
}

\linespread{1.15}

\title{Sterowanie teleskopu na montażu paralaktycznym}
\author{Paweł Szewczyk} 
\date{}

\pagestyle{fancy}
\fancyhead{}
\fancyfoot{}
\fancyfoot[RO, LE]{\thepage}
\renewcommand{\headrulewidth}{0pt}

\begin{document}

\maketitle

\tableofcontents

\chapter{Wprowadzenie do astronomii obserwacyjnej i astrofotografii}
\label{cha:wstep}

\section{Astrofotografia}

Astrofotografia to specyficzna dziedzina fotografii. Fotografując nocne niebo
napotkamy kilka kluczowych problemów nie występujących w klasycznej fotografii,
które czynią tę dziedzinę niełatwą.

Pierwszym z tych problemów jest niewielka jasność większości interesujących
obiektów na niebie. Od tej reguły są oczywiście wyjątki takie jak planety,
księżyc czy słońce. Jednak olbrzymią część interesujących obiektów stanowią tzw.
obiekty głębokiego nieba, czyli m.in. galaktyki, mgławice oraz gromady gwiazd.
Większość z nich jest na tyle ciemna, że niedostrzegalna gołym okiem nawet
w najbardziej korzystnych warunkach. Dlatego używane w astrofotografii sensory
światła muszą być bardzo czułe, a czasy naświetlania zdjęć dużo większe niż
w klasycznej fotografii.

Kolejnym problemem są niewielkie rozmiary kątowe fotografowanych obiektów. Tu
też są wyjątki, przykładowo droga mleczna, która zajmuje praktycznie całe nasze
niebo. Jest jednak wiele interesujących obiektów, których rozmiary kątowe
liczymy w minutach kątowych\footnote{Minuta kątowa - 1/60 stopnia}. Do robienia
takich zdjęć nie wystarczy zwykły obiektyw, potrzebne są większe i bardziej
zaawansowane układy optyczne o większych ogniskowych.

Ważny jest też fakt, że niebo, które fotografujemy, porusza się stale względem
nas. Ten pozorny dobowy ruch wszystkich ciał niebieskich wynika z ruchu
obrotowego ziemi, podobnie jak ruch słońca. Prędkość tego ruchu jest stała
i wynosi dokładnie jeden obrót na dobę gwiazdową\footnote{Doba gwiazdowa - okres
obrotu ziemi wokół własnej osi, wynoszący ok 86164s}.  O ile w klasycznej
fotografii taka prędkość nie stanowi problemu, to w fotografii nieba, biorąc pod
uwagę potrzebę długich czasów naświetlania, jest to istotny problem. Zrobienie
udanego zdjęcia wymaga, aby obiekt nie poruszał się przez czas naświetlania.
Biorąc dodatkowo pod uwagę długie ogniskowe teleskopu problem efektywnego
śledzenia ruchu ciał niebieskich okazuje się kluczowy w astrofotografii.

\section{Kamery używane w astrofotografii}

Zdjęcia o długich czasach ekspozycji stanowią duże wyzwanie dla kamer. Jednym
z głównych problemów jest szum zakłócający wynikowy obraz, przede wszystkim szum
termiczny. Dlatego używane w kamerach cyfrowych matryce muszą być szczególnie
dobrej jakości, a w bardziej profesjonalnych zastosowaniach są one chłodzone do
bardzo niskich temperatur (poniżej $-30\degree C$), aby zminimalizować szum
termiczny.

W astrofotografiii amatorskiej popularnym i ekonomicznym rozwiązaniem jest
stosowanie lustrzanek cyfrowych. Są one przystosowane do robienia zdjęć
o dowolnych czasach ekspozycji, a do tego łatwo dostępne i nie wymagające
żadnych modyfikacji. Wśród lustrzanej cyfrowych najbardziej popularne są
kolorowe matryce CMOS oraz CCD.

Dedykowane kamery astronomiczne pozwalają na dużo lepszą jakość zdjęć.
Największą popularność mają kamery korzystające z monochromatycznych matryc CCD.
Te urządzenia są jednak znacznie droższe od lustrzanek cyfrowych.

Do fotografii planet i księżyca często stosuje się zwykłe kamery internetowe,
ponieważ nie są w tym przypadku potrzebne duże czasy naświetlania, ale jak
najwięcej ekspozycji o krótkich czasach.

\section{Teleskopy}

W obserwacji nieba w zakresie światła widzialnego stosuje się dwa podstawowe
rodzaje układów optycznych: zwierciadlane (korzystające ze zjawiska odbicia
światła) i soczewkowe (korzystające ze zjawiska załamania, zwane też lunetami).
Istnieją także układy zwierające zarówno soczewki jak i zwierciadła (np.
teleskop Schmidta czy teleskop Maksutowa). 

Układy optyczne są zawsze obarczone wadami optycznymi:

\begin{itemize}

	\item Aberracja sferyczna to wada występująca w zwierciadłach
		i soczewkach sferycznych. Promienie przechodzące przez układ
		optyczny w różnej odległości od osi optycznej nie ogniskują się
		w jednym miejscu.
	
	\item Aberracja chromatyczna, występująca w lunetach, wynika z różnego
		kąta załamania światła o różnych długościach fali. Każda
		soczewka działa jak pryzmat, powodując rozszczepienie światła na
		składowe.

	\item Koma to wada polegająca na rozciągnięciu obrazu punktów nie
		leżących na osi optycznej. W efekcie gwiazdy leżące na granicy
		pola widzenia teleskopu przyjmują kształt przecinka. Koma
		występuje w zwierciadłach parabolicznych.

\end{itemize}

\section{Współrzędne sferyczne}

\begin{figure}
\label{fig_sfera}
\includesvg{sfera.svg}
\caption{Współrzędne na sferze niebieskiej}
\end{figure}

Rozpatrując pozycje i ruchy ciał niebieskich będziemy poruszali się na
\emph{sferze niebieskiej}. Sfera niebieska to pojęcie abstrakcyjne oznaczające
sferę o dowolnym promieniu i o środku w miejscu obserwacji.  Półprosta
poprowadzona z miejsca obserwacji, przecinająca dany obiekt, przecina sferę
niebieską w punkcie zwanym \emph{położeniem} tego obiektu na sferze niebieskiej
\cite{rybka}.

Promień sfery niebieskiej nie ma dla nas znaczenia jako dla obserwatorów.
Interesuje nas jedynie położenie na sferze. Dobrą analogią do tej sytuacji jest
określanie położenia na kuli ziemskiej za pomocą współrzędnych geograficznych:
długości i szerokości. Podobnie będziemy określali współrzędne na sferze
niebieskiej - za pomocą analogicznych dwóch miar kątów. Jedynym co musimy
ustalić jest położenie biegunów i środek układu współrzędnych.

W tej pracy korzystać będziemy głównie z układu \emph{równikowego}. Bieguny
znajdują się w nim w punktach przecięcia osi obrotu ziemi ze sferą niebieską.
Daje to tę podstawową zaletę, że ciała niebieskie mają niezmienną odległość od
tych punktów, więc w ciągu ich dobowego ruchu po sferze niebieskiej jedna ze
współrzędnych (odpowiadająca szerokości geograficznej) będzie niezmienna. Ta
współrzędna to \emph{deklinacja} oznaczana grecką literą $\delta$.

Podobnie jak w układzie współrzędnych geograficznych, na sferze niebieskiej
możemy wyznaczyć okrąg wielki będący \emph{równikiem niebieskim}, czyli zawarty
w płaszczyźnie prostopadłej do osi obrotu sfery niebieskiej. Punkty na równiku
mają zerową deklinację.

Ustalenie drugiej współrzędnej wymaga wybrania południka zerowego na niebie
(odpowiednika południka zerowego w układzie współrzędnych geograficznych).
Istnieją dwa rodzaje układu współrzędnych tego typu, tj. układu współrzędnych
równikowych.  Pierwszy z nich, układ \emph{równikowy godzinny}, za południk
zerowy obiera \emph{południk lokalny}, czyli południk przechodzący przez zenit.
Odpowiednikiem długości geograficznej jest w nim \emph{kąt godzinny}. W ciągu
dobowego ruchu sfery niebieskiej południk lokalny jest niezmienny względem
obserwatora, ale zmieniają się kąty godzinne ciał niebieskich. W drugim układzie
współrzędnych, \emph{równikowym równonocnym}, południk zerowy przebiega przez
punkt równonocy wiosennej, inaczej punkt barana oznaczany symbolem $\aries$.
Odpowiednikiem długości geograficznej jest w nim \emph{rektascensja} oznaczana
grecką literą $\alpha$ która jest stała i niezależna od miejsca obserwacji.
Układ równikowy równonocny jest najczęściej stosowany w atlasach nieba do
określania bezwzględnej pozycji ciał niebieskich.

Deklinację mierzymy zazwyczaj w stopniach i zawiera się w przedziale od
$-90\degree$ do $90\degree$. Rektascensję również możemy mierzyć w stopniach,
jednak zazwyczaj podaje się ją w mierze czasowej, od $0$ do $24h$.

\section{Montaż teleskopu}

W astronomii obserwacyjnej instrumenty obserwacyjne korzystają z dwóch typów
mocowania teleskopu (montaży): azymutalnego i paralaktycznego. Montaż azymutalny
jest prostszy w konstrukcji i w użyciu, korzysta bowiem z naturalnego układu
współrzędnych horyzontalnych. Takie montaże są jednak mało praktyczne
w astrofotografii amatorskiej. Montaż paralaktyczny jest natomiast bardzo
użyteczny w fotografowaniu nieba, ponieważ porusza się w opisanym wyżej układzie
równikowym. Oznacza to, że do śledzenia ruchu fotografowanych obiektów wystarczy
zmiana w tylko jednej osi. 

Takie montaże to zazwyczaj bardzo precyzyjne urządzenia. Załóżmy parametry
typowe dla amatorskiego zestawu do astrofotografii: ogniskowa teleskopu 750mm
i matryca lustrzanki cyfrowej o rozmiarach 22.2mm x 14.8mm. Zdjęcie będzie
obejmowało obszar na niebie o kątowych rozmiarach:

$$2*arc tg(d/2f)$$

Czyli dla powyższych parametrów ok. 1,7 x 1,3 stopnia. Zakładając rozdzielczość
finalnego zdjęcia rzędu 1000px, każdy piksel ma rozmiar rzędu kilku sekund
kątowych. Każde przesunięcie o 10'' jest widoczne na takim zdjęciu.

Aby uzyskać żądaną dokładność montaże astronomiczne zawierają wbudowane
przekładnie umożliwiające bardzo niewielkie przesunięcia, rzędu sekund
kątowych lub mniejsze. Ręczna korekcja pozycji na bieżąco jest oczywiście
niedokładna i wyjątkowo męcząca, biorąc pod uwagę czasy ekspozycji zdjęć.
Dlatego montaże wyposaża się w mechanizmy automatycznie obracające teleskopem.
W najprostszym scenariuszu wystarcza obrót w jednej osi (osi kąta godzinnego) ze
stałą prędkością. W efekcie obserwowany obiekt pozostaje w polu widzenia
teleskopu bez potrzeby ręcznej korekcji.

\section{Systemy GoTo}

Jednym z kolejnych praktycznych problemów przy wykonywaniu zdjęć niewielkich
obiektów głębokiego nieba jest ich wyszukiwanie na niebie. Można robić to
ręcznie naprowadzając oś optyczną teleskopu na wybrany punkt na niebie np.
z pomocą niewielkiej lunety dołączanej do głównego teleskopu (tzw. szukacz).
Kiedy jednak do głównego teleskopu dołączona jest kamera, trudno jest użyć go do
potwierdzenia, że ustawiliśmy teleskop poprawnie. Ciężko jest tu zastosować
techniki znane z obserwacji wizualnych.  Ręczne ustawianie teleskopu wymaga
w takiej sytuacji wiele doświadczenia i dobrej znajomości nieba, a zajmuje dużo
cennego czasu, który moglibyśmy poświęcić na ekspozycję zdjęcia.

Rozwiązaniem tego problemu jest system GoTo. Korzysta on z napędu w obu osiach
montażu do ustawiania go w żądanej pozycji. Po szybkiej kalibracji na jasnych
obiektach (np. gwiazdach) system jest gotowy do użycia.

\section{Zaawansowane metody śledzenia ruchu sfery niebieskiej}

Ostatnim usprawnieniem, o jakim będę wspomniał w tej pracy jest tzw.
\emph{guiding}.  Rozwiązuje on problem dokładnego śledzenia obiektów, jednak
w bardziej wyrafinowany sposób niż sterowanie napędem w jednej osi. Teoretycznie
powinien on działać wystarczająco dobrze, na drodze stają nam jednak dwa
problemy: niedokładne wykonanie przekładni i niedokładne ustawienie montażu.
Pierwszy problem powoduje efekt niejednostajnego ruchu montażu pomimo
jednostajnego obrotu śruby mikroruchów. Drugi problem zależy od wprawy
użytkownika montażu, który przed użyciem musi ustawić jego oś dokładnie tak, aby
przebiegała przez biegun niebieski. W praktyce to ustawienie nigdy nie jest
idealne i po odpowiednio długim czasie śledzenia następuje widoczne przesunięcie
w osi deklinacji.

Systemy guidingu polegają na zastosowaniu drugiego, mniejszego teleskopu
z czułą kamerą, przymocowanego do głównego teleskopu, do rejestrowania obrazu
nieba w czasie rzeczywistym. Ten obraz jest na bieżąco analizowany pod kątem
niepożądanych przesunięć, które są później korygowane przez odpowiedni moduł
elektroniczny.

W praktyce za analizę obrazu i korekcję ruchu odpowiada zazwyczaj zewnętrzny
komputer, sterownik teleskopu ma jedynie za zadanie umożliwić wprowadzenie
takich korekcji z zewnątrz.


\chapter{Założenia projektowanego rozwiązania}
\input{usecase.tex}
\section{Przegląd użytych komponentów sprzętowych}

\subsection{nrf51822}

Głównym modułem zaprojektowanego urządzenia jest nrf51822. Realizuje on
jednocześnie komunikację (Bluetooth) oraz logikę sterownika.

Moduł nrf51822 oparty jest o 32-bitowy układ ARM Cortex M0. Umożliwia
komunikację z użyciem Bluetooth Low Energy.

\subsection{A4988}

Komponentem odpowiedzialnym za sterowanie silnikami jest moduł A4988. Jest to
sterownik silników krokowych z możliwością pracy w trybie mikrokroków.

Napięcie zasilania silników: 8V - 35V

Maksymalne natężenie prądu: 2A

Możliwość pracy w trybach pełnego kroku, 1/2 kroku, 1/4 kroku, 1/8 kroku
i 1/16 kroku

Moduł posiada układ ograniczający prąd pobierany przez silnik, korzystający
z modulacji PWM. Limit prądu można ustawić za pomocą potencjometru wbudowanego
w układ.  Maksymalna wartość prądu wynosi:

$$I_{TripMax} = V_{ref}/(8*R_s)$$

Gdzie $R_s$ jest rezystancją czujnika wynoszącą dla użytego układu $0.05\ohm$
a $V_{ref}$ napięciem na pinie REF (możliwym do łatwego zmierzenia).

Ograniczenie prądu umożliwia również pracę z silnikami o niższym napięciu
nominalnym niż 8V.

\subsection{Silniki}

Sterownik pracuje z bipolarnymi silnikami krokowymi o nominalnych wartościach
prądu i napięcia mniejszych niż limit sterownika A4988: 2A i 35V. Przed użyciem
silników należy się upewnić co do limitu prądu ustawionego w sterowniku:
powinien być on równy wartości nominalnej prądu dla użytego silnika.

Silniki użyte do testów urządzenia to bipolarne silniki krokowe o nominalnych
wartościach napięcia i prądu odpowiednio 12V i 400mA.

Liczba kroków na pełen obrót użytego silnika krokowego jest konfigurowana w kodzie sterownika
jako MOTOR\_STEP\_RATIO i dla testowanych silników wynosi 200. Użycie innej
wartości (np 400 kroków dla niektórych silników) wymaga ponownej kompilacji
kodu sterownika.


\chapter{Sterowanie silnikami krokowymi}
\input{silniki.tex}

\chapter{Protokół komunikacji: Bluetooth Low Energy}
Użyta w projekcie technologia komunikacji to \emph{Bluetooth Low Energy} (BLE).
BLE jest zaprojektowane jako energooszczędna implementacja standardu Bluetooth
i pomimo użycia podobnej warstwy fizycznej nie jest kompatybilna z klasyczną
wersją, \emph{Bluetooth Basic Rate / Enchanced Data Rate} (BR/EDR). Obecnie
wiele urządzeń posiada wsparcie obu protokołów (tryb \emph{dual mode}).

Warstwa fizyczna BLE podobnie jak w przypadku BR/EDR korzysta z pasma radiowego
2.4GHz, jednak z mniejszą liczbą kanałów o większej szerokości. Wraz z warstwą
łącza (ang link layer) warstwa fizyczna jest często realizowana przez moduł
kontrolera (ang controller). Kontrolery bluetooth udostępniają interfejs HCI
(ang Host-Controller Interface) służący do komunikacji między warstwą łącza
a wyższymi warstwami. 

Kolejna warstwa stosu BLE to \emph{Logical Link Control and Adaptation Layer}
(L2CAP). Jest odpowiedzialna za segmentację danych i przekazywanie danych
o jakości usługi (QoS) do wyższych warstw.

Z L2CAP korzysta \emph{Attribute Protocol} (ATT). Definiuje on protokół wymiany
danych w architekturze klient-serwer oparty na \emph{atrybutach} (ang.
attribute).

\section{Protokół GAP}

Za odkrywanie dostępnych urządzeń i nawiązywanie połączeń odpowiedzialny jest
protokół GAP (\emph{Generic Access Profile}). Udostępnia też mechanizmy
bezpieczeństwa.

\subsection{Role urządzeń}

Specyfikacja GAP definiuje cztery role jakie mogą odgrywać urządzenia w procesie
komunikacji w trybie BLE:

\begin{itemize}

\item Urządzenie rozgłaszające (ang. Broadcaster)

\item Obserwator (ang. Observer)

\item Urządzenie peryferyjne (ang. Peripherial)

\item Urządzenie centralne (ang. Central)

\end{itemize}

\subsection{Proces rozgłaszania}

Urządzenia rozgłaszające wysyłają dane rozgłoszeniowe do wszystkich urządzeń
znajdujących się w zasięgu. Urządzenia pracujące jako obserwatorzy nasłuchują na
dane rozgłoszeniowe innych urządzeń.

W trakcie rozgłaszania danych urządzenie wysyła swój pakiet rozgłoszeniowy
w regularnych odstępach czasu. Odstęp między kolejnymi transmisjami pakietu może
być dowolną wielokrotnością $0.625ms$ między $20ms$ a $10.24ms$
\cite{bluetooth42}.

\subsection{Nawiązanie połączenia}

Urządzenie przyjmuje rolę urządzenia centralnego w momencie zainicjowania
połączenia z innym urządzeniem. Jeśli urządzenie zaakceptuje połączenie
z urządzeniem centralnym, przyjmuje rolę urządzenia peryferyjnego.

\section{Protokół GATT}

Protokół GATT porządkuje atrybuty ATT w hierarchiczną strukturę serwisów.

Nadrzędnym obiektem w hierarchii GATT jest \emph{profil} (ang. profile). Profile
składają się z jednego lub więcej \emph{serwisów} (ang. service).

Serwis złożony jest z \emph{charakterystyk} (ang. characterisics), które
reprezentują pojedyncze wartości w serwisie (analogicznie do pól struktury lub
klasy). Charakterystyka zawiera swoją wartość oraz opcjonalne metadane mówiące
o jej przeznaczeniu, uprawnieniach odczytu i zapisu.

\section{Opis profilu użytego w projekcie}

\begin{table}[t]

\begin{tabularx}{\linewidth}{|l|l|l|X|}

\hline UUID & Nazwa & Długość [B] & opis \\

\hline 0x1525 & CHAR\_POS & 8 & Pozycja w układzie współrzędnych równikowych \\

\hline 0x1526 & CHAR\_DEST & 8 & Pozycja docelowa dla systemu GoTo \\

\hline 0x1527 & CHAR\_MODE & 1 & Tryb pracy \\

\hline 0x1528 & CHAR\_REVERSE & 1 & Flaga odwracająca układ współrzędnych \\

\hline \end{tabularx}

\caption{Zestaw charakterystyk serwisu EQ-Driver}

\label{tab:eq-driver-service}

\end{table}

\begin{table}[t]

\begin{tabularx}{\linewidth}{|l|l|X|}

\hline Wartość & Nazwa &  opis \\

\hline 0x01 & MODE\_TRACKING & Tryb śledzenia \\

\hline 0x02 & MODE\_GOTO & Tryb GoTo \\

\hline 0x03 & MODE\_MANUAL & Tryb ręcznego sterowania \\

\hline 0xff & MODE\_OFF & Tryb bezczynności \\

\hline \end{tabularx}

\caption{Tryby pracy sterownika z wartościami kodów zapisanych
w charakterystyce}

\label{tab:eq-driver-modes}

\end{table}

Na potrzeby projektu powstał profil GATT umożliwiający komunikację ze
sterownikiem teleskopu. Składa się on z jednego serwisu posiadającego kilka
charakterystyk. Tabela \ref{tab:eq-driver-service} opisuje serwis EQ-Driver.

\subsection{Kodowanie pozycji}

Pozycja w układzie współrzędnych równikowych zapisana w charakterystykach
serwisu kodowana jest na 8 bajtach. Pierwsze 4 bajty pozycji zawierają
informację o rektascensji, pozostałe 4 bajty o deklinacji.

Każda z tych wartości kodowana jest w formacie Little Endian i podana
w sekundach kątowych, czyli $1/1296000$ części łuku.

\subsection{Kodowanie trybu pracy}

Tabela \ref{tab:eq-driver-modes} opisuje wartości możliwe do zapisania
w charakterystyce odpowiadającej trybowi pracy. Jest on kodowany na jednym
bajcie i może przyjmować jedną z czterech wartości podanych w tabeli.

\subsection{Obsługa nieprawidłowych wartości}

Zapisanie nieprawidłowej wartości do charakterystyki trybu pracy nie powiedzie
się i ta nie ulegnie zmianie.

Zapisanie nieprawidłowej wartości pozycji spowoduje przekształcenie jej
w wartość poprawną.

\section{Implementacja}

\subsection{Inicjalizacja stosu Bluetooth}

Za zainicjowanie BLE z odpowiednimi parametrami odpowiada funkcja
\emph{ble\_init()}. Konfiguruje ona podsystem \emph{SoftDevice} i ustawia
parametry rozgłaszania i połączeń.

Konfiguracja parametrów BLE znajduje się w pliku \emph{eq\_ble.h}.

\begin{table}[t]

\begin{tabularx}{\linewidth}{|l|X|l|}

\hline Parametr & Opis & Wartość \\

\hline ADV\_INCLUDE\_APPEARANCE & Flaga oznaczająca dołączenie charakterystyki
\emph{appearance} & 0 \\

\hline ADV\_FAST\_ENABLED & Umożliwienie trybu szybkiego rozgłaszania & 1 \\

\hline ADV\_FAST\_INTERVAL & Interwał rozgłaszania w trybie szybkiego
rozgłaszania & $32.5ms$ \\

\hline ADV\_FAST\_TIMEOUT & Czas trwania trybu szybkiego rozgłaszania & $120s$
\\

\hline DEVICE\_NAME & Rozgłaszana nazwa urządzenia & ''EQ-Driver'' \\

\hline \end{tabularx}

\caption{Konfiguracja rozgłaszania}

\label{tab:parametry-rozglaszania}

\end{table}

Parametry rozgłaszania opisuje \ref{tab:parametry-rozglaszania}.

SoftDevice pozwala na konfigurację czterech różnych trybów rozgłaszania:

\begin{itemize}

\item Bezpośredni (Directed)- po rozłączeniu aplikacja próbuje połączyć się
ponownie z urządzeniem, z którym utraciła połączenie

\item Szybki (fast) - Aplikacja rozgłasza dane w krótkich interwałach

\item Wolny (slow) - Aplikacja rozgłasza dane w długich interwałach aby
oszczędzać energię

\item Idle - Aplikacja nie rozgłasza danych

\end{itemize}

Aplikacja włącza kolejne tryby na określony czas badź do czasu nawiązania
połączenia. Jeśli żadne urządzenie nie zainicjuje połączenia w trybie szybkim,
rozpocznie się rozgłaszanie z dłuższym interwałem obniżając zużycie energii.

\begin{table}[t]

\begin{tabularx}{\linewidth}{|l|X|l|}

\hline Parametr & Opis & Wartość \\

\hline MIN\_CONN\_INTERVAL & Dolne ograniczenie czasu pomiędzy transmisjami
danych & $125ms$ \\

\hline MAX\_CONN\_INTERVAL & Górne ograniczenie czasu pomiędzy transmisjami
danych & $250ms$ \\

\hline SLAVE\_LATENCY & Liczba kolejnych żądań na które urządzenie może nie
odpowiadać & 0 \\

\hline CONN\_SUP\_TIMEOUT & Czas od ostatniego dostarczenia danych po którym
połączenie jest uważane za utracone & $40s$ \\

\hline \end{tabularx}

\caption{Konfiguracja połączeń}

\label{tab:parametry-polaczenia}

\end{table}

Po nawiązaniu połączenia urządzenie centralne będzie żądać od urządzenia
peryferyjnego danych regularnie. Parametry MIN\_CONN\_INTERVAL
i MAX\_CONN\_INTERVAL przekazywane są urządzeniu centralnemu i ustalają
ograniczenie czasu pomiędzy kolejnymi żądaniami.

Parametr SLAVE\_LATENCY umożliwia urządzeniu peryferyjnemu na nieodpowiadanie na
żądania urządzenia centralnego $n$ razy, co pozwala mu na dłuższy okres
bezczynności, kiedy nie ma nowych danych do wysłania. Ustawienie tej wartości na
0 wyłącza tę funkcjonalność.

Parametr CONN\_SUP\_TIMEOUT definiuje maksymalny czas od ostatniej transmisji
danych. Po tym czasie połączenie jest uważane za utracone i urządzenie może
próbować połączyć się z nim ponownie.

\subsection{Obsługa zdarzeń protokołu GAP}

TODO

\subsection{Obsługa zdarzeń protokołu GATT}

TODO


\chapter{Interfejs użytkownika: projekt aplikacji mobilnej}

\chapter{Testy}

\chapter{Podsumowanie}

\begin{thebibliography}{9}

\bibitem{rybka} Eugeniusz Rybka, \emph{Astronomia Ogólna}

\bibitem{bluetooth42} \emph{Specification of the Bluetooth System, v4.2}
	
\bibitem{przepiorkowski} Jacek Przepiórkowski, \emph{Silniki elektryczne
	w praktyce elektronika} \\ Wydawnictwo BTC Warszawa 2007

\bibitem{nrf51-manual} Nordic Semiconductor, \emph{nRF51 Series Reference Manual
	Version 2.1}

\end{thebibliography}


\end{document}
