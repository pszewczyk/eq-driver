Użyta w projekcie technologia komunikacji to \emph{Bluetooth Low Energy} (BLE).
BLE jest zaprojektowane jako energooszczędna implementacja standardu Bluetooth
i pomimo użycia podobnej warstwy fizycznej nie jest kompatybilna z klasyczną
wersją, \emph{Bluetooth Basic Rate / Enchanced Data Rate} (BR/EDR). Obecnie
wiele urządzeń posiada wsparcie obu protokołów (tryb \emph{dual mode}).

Warstwa fizyczna BLE podobnie jak w przypadku BR/EDR korzysta z pasma radiowego
2.4GHz, jednak z mniejszą liczbą kanałów o większej szerokości. Wraz z warstwą
łącza (ang link layer) warstwa fizyczna jest często realizowana przez moduł
kontrolera (ang controller). Kontrolery bluetooth udostępniają interfejs HCI
(ang Host-Controller Interface) służący do komunikacji między warstwą łącza
a wyższymi warstwami. 

Kolejna warstwa stosu BLE to \emph{Logical Link Control and Adaptation Layer}
(L2CAP). Jest odpowiedzialna za segmentację danych i przekazywanie danych
o jakości usługi (QoS) do wyższych warstw.

Z L2CAP korzysta \emph{Attribute Protocol} (ATT). Definiuje on protokół wymiany
danych w architekturze klient-serwer oparty na \emph{atrybutach} (ang.
attribute).

\section{Protokół GAP}

Za odkrywanie dostępnych urządzeń i nawiązywanie połączeń odpowiedzialny jest
protokół GAP (\emph{Generic Access Profile}). Udostępnia też mechanizmy
bezpieczeństwa.

\subsection{Role urządzeń}

Specyfikacja GAP definiuje cztery role jakie mogą odgrywać urządzenia w procesie
komunikacji w trybie BLE:

\begin{itemize}

\item Urządzenie rozgłaszające (ang. Broadcaster)

\item Obserwator (ang. Observer)

\item Urządzenie peryferyjne (ang. Peripherial)

\item Urządzenie centralne (ang. Central)

\end{itemize}

\subsection{Proces rozgłaszania}

Urządzenia rozgłaszające wysyłają dane rozgłoszeniowe do wszystkich urządzeń
znajdujących się w zasięgu. Urządzenia pracujące jako obserwatorzy nasłuchują na
dane rozgłoszeniowe innych urządzeń.

W trakcie rozgłaszania danych urządzenie wysyła swój pakiet rozgłoszeniowy
w regularnych odstępach czasu. Odstęp między kolejnymi transmisjami pakietu może
być dowolną wielokrotnością $0.625ms$ między $20ms$ a $10.24ms$
\cite{bluetooth42}.

\subsection{Nawiązanie połączenia}

Urządzenie przyjmuje rolę urządzenia centralnego w momencie zainicjowania
połączenia z innym urządzeniem. Jeśli urządzenie zaakceptuje połączenie
z urządzeniem centralnym, przyjmuje rolę urządzenia peryferyjnego.

\section{Protokół GATT}

Protokół GATT (ang. Generic Attributes) porządkuje atrybuty ATT w hierarchiczną
strukturę serwisów. GATT pozwala na zdefiniowanie profilu odpowiadającego
funkcjonalności urządzenia.

Nadrzędnym obiektem w hierarchii GATT jest \emph{profil} (ang. profile). Profile
GATT mają podobne zadanie do tradycyjnych profili Bluetooth: definiują wymagania
dla aplikacji mających spełniać typowe zadania. Specyfikacja protokołu GATT
również definiuje zestaw profili dla typowych zastosowań takich jak np. czujniki
biomedyczne (pulsometry, czujniki ciśnienia krwi itp.). GATT umożliwia jednak
łatwe zdefiniowanie własnego profilu w przypadkach nieobjętych specyfikacją
protokołu.

Profile składają się z jednego lub więcej \emph{serwisów} (ang. service). Serwis
reprezentuje spójny zestaw funkcjonalności dostarczanych przez urządzenie.

Serwis złożony jest z \emph{charakterystyk} (ang. characterisics), które
reprezentują pojedyncze wartości w serwisie (analogicznie do pól struktury lub
klasy). Charakterystyka zawiera swoją wartość oraz opcjonalne metadane mówiące
o jej przeznaczeniu, uprawnieniach odczytu i zapisu.

Z punktu widzenia warstwy aplikacji protokół GATT pozwala na odczytywanie
i zapisywanie charakterystyk zebranych w strukturę serwisów i profili.

\section{Opis profilu użytego w projekcie}

\begin{table}[t]

\begin{tabularx}{\linewidth}{|l|l|l|X|}

\hline UUID & Nazwa & Długość [B] & opis \\

\hline 0x1525 & CHAR\_POS & 8 & Pozycja w układzie współrzędnych równikowych \\

\hline 0x1526 & CHAR\_DEST & 8 & Pozycja docelowa dla systemu GoTo \\

\hline 0x1527 & CHAR\_MODE & 1 & Tryb pracy \\

\hline 0x1528 & CHAR\_REVERSE & 1 & Flaga odwracająca układ współrzędnych \\

\hline \end{tabularx}

\caption{Zestaw charakterystyk serwisu EQ-Driver}

\label{tab:eq-driver-service}

\end{table}

\begin{table}[t]

\begin{tabularx}{\linewidth}{|l|l|X|}

\hline Wartość & Nazwa &  opis \\

\hline 0x01 & MODE\_TRACKING & Tryb śledzenia \\

\hline 0x02 & MODE\_GOTO & Tryb GoTo \\

\hline 0x03 & MODE\_MANUAL & Tryb ręcznego sterowania \\

\hline 0xff & MODE\_OFF & Tryb bezczynności \\

\hline \end{tabularx}

\caption{Tryby pracy sterownika z wartościami kodów zapisanych
w charakterystyce}

\label{tab:eq-driver-modes}

\end{table}

Na potrzeby projektu powstał profil GATT umożliwiający komunikację ze
sterownikiem teleskopu. Składa się on z jednego serwisu posiadającego kilka
charakterystyk. Tabela \ref{tab:eq-driver-service} opisuje serwis EQ-Driver.

\paragraph{Kodowanie pozycji} Pozycja w układzie współrzędnych równikowych
zapisana w charakterystykach serwisu kodowana jest na 8 bajtach. Pierwsze
4 bajty pozycji zawierają informację o rektascensji, pozostałe 4 bajty
o deklinacji.

Każda z tych wartości kodowana jest w formacie Little Endian i podana
w sekundach kątowych, czyli $1/1296000$ części łuku.

\paragraph{Kodowanie trybu pracy} Tabela \ref{tab:eq-driver-modes} opisuje
wartości możliwe do zapisania w charakterystyce odpowiadającej trybowi pracy.
Jest on kodowany na jednym bajcie i może przyjmować jedną z czterech wartości
podanych w tabeli.

\paragraph{Obsługa nieprawidłowych wartości} Zapisanie nieprawidłowej wartości
do charakterystyki trybu pracy nie powiedzie się i ta nie ulegnie zmianie.

Zapisanie nieprawidłowej wartości pozycji spowoduje przekształcenie jej
w wartość poprawną za pomocą operacji modulo.

\section{Implementacja}

Biblioteka \emph{SoftDevice} implementuje stos protokołu Bluetooth low energy.
Warstwa aplikacyjna odpowiedzialna jest za poprawne skonfigurowanie protokołu
i obsługę zdarzeń związanych z komunikacją.

\subsection{Inicjalizacja stosu Bluetooth}

Za zainicjowanie BLE z odpowiednimi parametrami odpowiada funkcja
\emph{ble\_init()}. Konfiguruje ona podsystem \emph{SoftDevice} i ustawia
parametry rozgłaszania i nawiązywanych połączeń.

Konfiguracja parametrów BLE znajduje się w pliku \emph{eq\_ble.h}.

\begin{table}[t]

\begin{tabularx}{\linewidth}{|l|X|l|}

\hline Parametr & Opis & Wartość \\

\hline ADV\_INCLUDE\_APPEARANCE & Flaga oznaczająca dołączenie charakterystyki
\emph{appearance} & 0 \\

\hline ADV\_FAST\_ENABLED & Umożliwienie trybu szybkiego rozgłaszania & 1 \\

\hline ADV\_FAST\_INTERVAL & Interwał rozgłaszania w trybie szybkiego
rozgłaszania & $32.5ms$ \\

\hline ADV\_FAST\_TIMEOUT & Czas trwania trybu szybkiego rozgłaszania & $120s$
\\

\hline DEVICE\_NAME & Rozgłaszana nazwa urządzenia & ''EQ-Driver'' \\

\hline \end{tabularx}

\caption{Konfiguracja rozgłaszania}

\label{tab:parametry-rozglaszania}

\end{table}

Parametry rozgłaszania opisuje \ref{tab:parametry-rozglaszania}.

SoftDevice pozwala na konfigurację czterech różnych trybów rozgłaszania:

\begin{itemize}

\item Bezpośredni (Directed)- po rozłączeniu aplikacja próbuje połączyć się
ponownie z urządzeniem, z którym utraciła połączenie

\item Szybki (fast) - Aplikacja rozgłasza dane w krótkich interwałach

\item Wolny (slow) - Aplikacja rozgłasza dane w długich interwałach aby
oszczędzać energię

\item Idle - Aplikacja nie rozgłasza danych

\end{itemize}

Aplikacja włącza kolejne tryby na określony czas bądź do czasu nawiązania
połączenia. Jeśli żadne urządzenie nie zainicjuje połączenia w trybie szybkim,
po skonfigurowanym czasie rozpocznie się rozgłaszanie z dłuższym interwałem
obniżając zużycie energii, po czym aplikacja wejdzie w tryb uśpienia po
przekroczeniu kolejnego ze skonfigurowanych czasów.

W projektowanym systemie aktywny jest tylko tryb szybkiego rozgłaszania.

\begin{table}[t]

\begin{tabularx}{\linewidth}{|l|X|l|}

\hline Parametr & Opis & Wartość \\

\hline MIN\_CONN\_INTERVAL & Dolne ograniczenie czasu pomiędzy transmisjami
danych & $125ms$ \\

\hline MAX\_CONN\_INTERVAL & Górne ograniczenie czasu pomiędzy transmisjami
danych & $250ms$ \\

\hline SLAVE\_LATENCY & Liczba kolejnych żądań na które urządzenie może nie
odpowiadać & 0 \\

\hline CONN\_SUP\_TIMEOUT & Czas od ostatniego dostarczenia danych po którym
połączenie jest uważane za utracone & $40s$ \\

\hline \end{tabularx}

\caption{Konfiguracja połączeń}

\label{tab:parametry-polaczenia}

\end{table}

Po nawiązaniu połączenia urządzenie centralne będzie żądać od urządzenia
peryferyjnego danych regularnie. Parametry MIN\_CONN\_INTERVAL
i MAX\_CONN\_INTERVAL przekazywane są urządzeniu centralnemu i ustalają
ograniczenie czasu pomiędzy kolejnymi żądaniami.

Parametr SLAVE\_LATENCY umożliwia urządzeniu peryferyjnemu na nieodpowiadanie na
żądania urządzenia centralnego $n$ razy, co pozwala mu na dłuższy okres
bezczynności, kiedy nie ma nowych danych do wysłania. Ustawienie tej wartości na
0 wyłącza tę funkcjonalność.

Parametr CONN\_SUP\_TIMEOUT definiuje maksymalny czas od ostatniej transmisji
danych. Po tym czasie połączenie jest uważane za utracone i urządzenie może
próbować połączyć się z nim ponownie.

\subsection{Obsługa zdarzeń protokołu GAP}

\subsection{Obsługa zdarzeń protokołu GATT}

TODO
