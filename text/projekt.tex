Główną część systemu stanowi sterownik napędu teleskopu. Odpowiada on za
sterowanie silnikami zamocowanymi do montażu i realizuje opisane w wymaganiach
systemu tryby pracy. Dostarcza również protokół komunikacji oparty o Bluetooth
Low Energy.

\section{nRF51822}

Sterownik korzysta modułu Core51822, opartego o układ nRF51822. Realizuje on
zarówno logikę oraz komunikację z urządzeniami tworzącymi interfejs użytkownika.

W skład modułu wchodzą:

\begin{itemize}

	\item 32 bitowy procesor ARM Cortex-M0.

	\item Pamięć flash o pojemności 256kB.

	\item Pamięć RAM o pojemności 32kB.

	\item Interfejs SWD pozwalający na debugowanie programu działającego na
		urządzeniu oraz jego aktualizację.

	\item Antena z modułem radiowym pracującym w paśmie ISM od 2.4GHz do
2.483GHz pozwalającym na implementację popularnych protokołów komunikacji takich
		jak Bluetooth.

	\item 29 pinów GPIO

\end{itemize}

\subsection{Zegary i liczniki}

W układach z serii nRF51 wykorzystywane są dwa główne zegary: o niskiej
częstotliwości (LFCLK) oraz o wysokiej częstotliwości (HFCLK). Zegar LFCLK
pracuje z częstotliwością 32.768kHz, natomiast zegar HFCLK z częstotliwością
16MHz.

Moduł nRF51822 udostępnia licznik count-up (zliczający w górę) korzystający
z zegara HFCLK. Dodatkowa warstwa abstrakcji dostarczona w bibliotekach SDK
umożliwia użycie wielu abstrakcyjnych liczników o różnych częstotliwościach.

\subsection{SoftDevice}

Warstwę abstrakcji między sprzętem a kodem aplikacji stanowi moduł programowy
SoftDevice. Składa się on z zestawu bibliotek realizujących komunikację
bezprzewodową i operacje wejścia/wyjścia. Jego głównym zadaniem w projekcie jest
implementacja protokołu Bluetooth.

Biblioteki dostarczone przez SoftDevice są prekompilowane i udostępniają
interfejs dla kodu aplikacji.

SoftDevice składa się z trzech kompontentów. Są to:

\begin{itemize}

	\item Biblioteka SoC udostępniająca interfejs programowy dla aplikacji.

	\item Menadżer SoftDevice będący interfejsem pozwalającym na
		konfigurację samego modułu SoftDevice, w tym włączanie
		i wyłączanie go.

	\item Stos protokołów - implementacja protokołów wspieranych przez
		SoftDevice.

\end{itemize}

W projekcie użyta została wersja S130 v2.0.1 dostarczona z SDK w wersji v12.2.0.

\subsection{Mapa pamięci}

Przestrzeń adresowa pamięci modułu dzieli się na bloki o różnym przeznaczeniu.

Blok pamięci o najniższych adresach zawiera kod programu. W przypadku
zastosowania oprogramowania SoftDevice w pamięci programu wydzielony jest osobny
blok przeznaczony na kod tego podsystemu.

Podobny podział występuje w pamięci RAM urządzenia. Rysunek \ref{rys:memory-map}
przedstawia podział pamięci.

Przedstawiony podział pamięci modułu wykorzystuje skrypt linkera używany przy
budowaniu całości oprogramowania sterownika.

\subsection{BLE400}

W procesie implementacji oprogramowania sterownika przydatna okazała się płytka
deweloperska BLE400. Zbiera ona kilka istotnych w tworzeniu i testowaniu systemu
elementów:

\begin{itemize}

\item Gniazdo na moduł Core51822 o odległości pinów 2mm.

\item Gniazdo na baterię zasilającą moduł.

\item Układ CP2102 realizujący komunikację między interfejsami szeregowymi USB
i UART.

\item Układ zasilania pozwalający na zasilanie podłączonego modułu przez USB.

\item Zestaw diod LED i przycisków podłączonych do konkretnych pinów modułu.

\end{itemize}

\section{Pozostałe komponenty sprzętowe}

Za generowanie impulsów sterujących silnikami odpowiadają dwa sterowniki A4988.

\section{Projekt układu elektronicznego}

\subsection{Wejścia i wyjścia}

\begin{table}[h]

\begin{tabularx}{\linewidth}{|l|l|X|}

\hline Nazwa & Numer & Opis \\

	\hline RA-DIR & 30 & Kierunek silnika osi rektascensji (osi kąta
	godzinnego) \\

	\hline RA-EN & 0 & Wyjście włączające / wyłączające silnik osi
	rektascensji \\

	\hline RA-STEP & 1 & Wyjście zegara sterującego silnikiem osi
	rektascensji \\
	
	\hline RA-MS1 & 2 & Pierwszy bit wyboru trybu pracy sterownika silnika
	osi rektascensji \\
	
	\hline RA-MS2 & 4 & Drugi bit wyboru trybu pracy sterownika silnika osi
	rektascensji \\
	
	\hline RA-MS2 & 3 & Trzeci bit wyboru trybu pracy sterownika silnika osi
	rektascensji \\
	
	\hline DEC-DIR & 9 & Kierunek silnika osi deklinacji \\

	\hline DEC-EN & 10 & Wyjście włączające / wyłączające silnik osi
	deklinacji \\

	\hline DEC-STEP & 7 & Wyjście zegara sterującego silnikiem osi
	deklinacji \\
	
	\hline DEC-MS1 & 8 & Pierwszy bit wyboru trybu pracy sterownika silnika
	osi deklinacji \\
	
	\hline DEC-MS2 & 6 & Drugi bit wyboru trybu pracy sterownika silnika osi
	 deklinacji\\
	
	\hline DEC-MS2 & 5 & Trzeci bit wyboru trybu pracy sterownika silnika
	osi deklinacji \\
	
	\hline LED1 & 28 & Dioda LED oznaczająca tryb śledzenia\\
	
	\hline LED2 & 24 & Dioda LED oznaczająca tryb GoTo \\

	\hline LED3 & 22 & Dioda LED oznaczająca tryb ręczny \\
	
	\hline RX & 17 & Pin RX portu szeregowego \\

	\hline TX & 18 & Pin TX portu szeregowego \\ 

	\hline CTS & 20 & Pin CTS portu szeregowego \\

	\hline RTS  & 19 & Pin RTS portu szeregowego \\

	\hline RA+ & 15 & Wejście sterujące w osi rektascensji w górę \\

	\hline RA- & 13 & Wejście sterujące w osi rektascensji w dół \\

	\hline DEC+ & 16 & Wejście sterujące w osi deklinacji w górę \\

	\hline DEC- & 14 & Wejście sterujące w osi deklinacji w dół \\

\hline \end{tabularx}

\caption{Funkcje pinów Core51822}

\label{tab:custom-pinout}

\end{table}

Do poszczególnych pinów modułu Core51822 przypisane są funkcje opisane w tabeli
\ref{tab:custom-pinout}.

\section{Architektura oprogramowania}

\subsection{Diagram stanów}

Na podstawie założeń dotyczących trybów pracy całego systemu opisanych
w rozdziale \ref{cha:usecase} można wyróżnić odpowiadające im stany w jakich
może znajdować się projektowany moduł sterownika:

\begin{itemize}

	\item Śledzenie - stan, w którym sterownik realizuje tryb śledzenia.
		W tym stanie sterownik dokonuje poprawek zgodnie z ruchem nieba.

	\item Stan GoTo, w którym sterownik realizuje zmianę wskazywanego przez
		oś optyczną teleskopu położenia, co sprowadza się do
		wygenerowania sygnału obracającego silniki montażu o ustalone
		wartości przesunięć kątowych.

	\item Stan sterowania ręcznego, w którym sterownik pozwala użytkownikowi
		na ręczne zmiany przesunięć kątowych montażu. 

	\item Sterownik wyłączony - stan, w którym sterownik nie steruje żadnym
		silnikiem.

\end{itemize}

Możliwe są przejścia między każdą parą stanów. Dla zwiększenia czytelności można
wprowadzić pomocniczy stan: \emph{konfiguracja trybu pracy}.  W rzeczywistości
sterownik może znajdować się w takim stanie, ponieważ zmiana stanu logicznego
nie jest natychmiastowa i wymaga zmiany konfiguracji sprzętu, co trwa pewną
ilość czasu.

\begin{figure}

	\includegraphics[width=\linewidth]{state-diagram}

	\caption{Diagram stanów sterownika teleskopu}
	
	\label{rys:state-diagram}

\end{figure}

Pomiędzy stanem GoTo a śledzeniem następuje automatyczne przejście, jeśli cel
trybu GoTo zostanie osiągnięty. Pozostałe przejścia na diagramie stanów
wymuszone są przez użytkownika.

Rysunek \ref{rys:state-diagram} pokazuje diagram stanów sterownika.

\section{Sterowanie silnikami}

Wykorzystując udostępniony przez SDK interfejs liczników (\emph{app\_timer})
sterownik generuje na wyjściach RA-STEP i DEC-STEP sygnały prostokątne, aby
obracać silnikami z prędkością odpowiednią dla stanu, w jakim się znajduje. Te
wyjścia podpięte są do wejść STEP modułów A4988.

Zależnie od stanu wykorzystywane są różne liczniki. Program sterownika definiuje
trzy liczniki:

\begin{lstlisting}

APP_TIMER_DEF(sky_movement_timer_id);
APP_TIMER_DEF(goto_timer_id);
APP_TIMER_DEF(manual_timer_id);

\end{lstlisting}

Licznik sky\_movement\_timer jest odpowiedzialny za śledzenie ruchu nieba
i aktywny jest zawsze. Jeśli sterownik znajduje się w stanie innym niż śledzenie
jest on używany do zmiany obecnego położenia o kąt przesunięcia nieba. W stanie
śledzenia jest on wykorzystywany do generowania sygnału sterującego silnikami,
aby położenie było zachowywane.

Pozostałe dwa liczniki aktywne są tylko w odpowiadających sobie trybach pracy.

Liczniki generują przerwania obsługiwane przez odpowiadające różnym trybom pracy
funkcje:

\begin{lstlisting}

static void sky_movement_timer_handler(void *p_context);
static void manual_timer_handler(void *p_context);
static void goto_timer_handler(void *p_context);

\end{lstlisting}

\section{Wsparcie dla zewnętrznych systemów śledzenia}

Dostępne na rynku systemy wspierające śledzenie ruchu sfery niebieskiej opierają
się na różnych standardach. Jednym z popularnych, jednak nieoficjalnych
standardów jest ST-4. Jest on wspierany przez projektowany system.

\begin{table}[h]

\begin{tabularx}{\linewidth}{|l|X|}

\hline Numer & Opis \\
	
	\hline 1 & Nieużywany \\
	
	\hline 2 & Masa \\

	\hline 3 & RA+ - zwiększenie rektascensji \\

	\hline 4 & DEC+ - zwiększenie deklinacji \\

	\hline 5 & DEC- - zmniejszenie deklinacji \\

	\hline 6 & RA- - Zmniejszenie rektascensji \\

\hline \end{tabularx}

\caption{Opis pinów gniazda ST-4}

\label{tab:st4-pinout}

\end{table}

W trybie śledzenia sterownik używa wejść RA+, RA-, DEC+ i DEC- do korekcji
pozycji montażu. Te wejścia są też wyprowadzone w postaci gniazda RJ12, o które
oparty jest standard ST-4. Tabela \ref{tab:st4-pinout} opisuje kolejne piny
gniazda.

\section{Port szeregowy}

%TODO

\section{Protokół komunikacji}

Biblioteka \emph{SoftDevice} implementuje stos protokołu Bluetooth low energy.
Warstwa aplikacyjna odpowiedzialna jest za poprawne skonfigurowanie protokołu
i obsługę zdarzeń związanych z komunikacją.

\subsection{Inicjalizacja stosu Bluetooth}

Za zainicjowanie BLE z odpowiednimi parametrami odpowiada funkcja
\emph{ble\_init()}. Konfiguruje ona podsystem \emph{SoftDevice} i ustawia
parametry rozgłaszania i nawiązywanych połączeń.

Konfiguracja parametrów BLE znajduje się w pliku \emph{eq\_ble.h}.

\begin{table}[t]

\begin{tabularx}{\linewidth}{|l|X|l|}

\hline Parametr & Opis & Wartość \\

\hline ADV\_INCLUDE\_APPEARANCE & Flaga oznaczająca dołączenie charakterystyki
\emph{appearance} & 0 \\

\hline ADV\_FAST\_ENABLED & Umożliwienie trybu szybkiego rozgłaszania & 1 \\

\hline ADV\_FAST\_INTERVAL & Interwał rozgłaszania w trybie szybkiego
rozgłaszania & $32.5ms$ \\

\hline ADV\_FAST\_TIMEOUT & Czas trwania trybu szybkiego rozgłaszania & $120s$
\\

\hline DEVICE\_NAME & Rozgłaszana nazwa urządzenia & ''EQ-Driver'' \\

\hline \end{tabularx}

\caption{Konfiguracja rozgłaszania}

\label{tab:parametry-rozglaszania}

\end{table}

Parametry rozgłaszania opisuje \ref{tab:parametry-rozglaszania}.

SoftDevice pozwala na konfigurację czterech różnych trybów rozgłaszania:

\begin{itemize}

\item Bezpośredni (Directed)- po rozłączeniu aplikacja próbuje połączyć się
ponownie z urządzeniem, z którym utraciła połączenie

\item Szybki (fast) - Aplikacja rozgłasza dane w krótkich interwałach

\item Wolny (slow) - Aplikacja rozgłasza dane w długich interwałach aby
oszczędzać energię

\item Idle - Aplikacja nie rozgłasza danych

\end{itemize}

Aplikacja włącza kolejne tryby na określony czas bądź do czasu nawiązania
połączenia. Jeśli żadne urządzenie nie zainicjuje połączenia w trybie szybkim,
po skonfigurowanym czasie rozpocznie się rozgłaszanie z dłuższym interwałem
obniżając zużycie energii, po czym aplikacja wejdzie w tryb uśpienia po
przekroczeniu kolejnego ze skonfigurowanych czasów.

W projektowanym systemie aktywny jest tylko tryb szybkiego rozgłaszania.

\begin{table}[t]

\begin{tabularx}{\linewidth}{|l|X|l|}

\hline Parametr & Opis & Wartość \\

\hline MIN\_CONN\_INTERVAL & Dolne ograniczenie czasu pomiędzy transmisjami
danych & $125ms$ \\

\hline MAX\_CONN\_INTERVAL & Górne ograniczenie czasu pomiędzy transmisjami
danych & $250ms$ \\

\hline SLAVE\_LATENCY & Liczba kolejnych żądań na które urządzenie może nie
odpowiadać & 0 \\

\hline CONN\_SUP\_TIMEOUT & Czas od ostatniego dostarczenia danych po którym
połączenie jest uważane za utracone & $40s$ \\

\hline \end{tabularx}

\caption{Konfiguracja połączeń}

\label{tab:parametry-polaczenia}

\end{table}

Po nawiązaniu połączenia urządzenie centralne będzie żądać od urządzenia
peryferyjnego danych regularnie. Parametry MIN\_CONN\_INTERVAL
i MAX\_CONN\_INTERVAL przekazywane są urządzeniu centralnemu i ustalają
ograniczenie czasu pomiędzy kolejnymi żądaniami.

Parametr SLAVE\_LATENCY umożliwia urządzeniu peryferyjnemu na nieodpowiadanie na
żądania urządzenia centralnego $n$ razy, co pozwala mu na dłuższy okres
bezczynności, kiedy nie ma nowych danych do wysłania. Ustawienie tej wartości na
0 wyłącza tę funkcjonalność.

Parametr CONN\_SUP\_TIMEOUT definiuje maksymalny czas od ostatniej transmisji
danych. Po tym czasie połączenie jest uważane za utracone i urządzenie może
próbować połączyć się z nim ponownie.

\subsection{Obsługa zdarzeń protokołu GAP}

\subsection{Obsługa zdarzeń protokołu GATT}

\section{Testy sterownika}
