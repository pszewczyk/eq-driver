\label{cha:wstep}

\section{Astrofotografia}

Astrofotografia to specyficzna dziedzina fotografii. Fotografując nocne niebo
napotkamy kilka kluczowych problemów nie występujących w klasycznej fotografii,
które czynią tę dziedzinę niełatwą.

Pierwszym z tych problemów jest niewielka jasność większości interesujących
obiektów na niebie. Od tej reguły są oczywiście wyjątki takie jak planety,
księżyc czy słońce. Jednak olbrzymią część interesujących obiektów stanowią tzw.
obiekty głębokiego nieba, czyli m.in. galaktyki, mgławice oraz gromady gwiazd.
Większość z nich jest na tyle ciemna, że niedostrzegalna gołym okiem nawet
w najbardziej korzystnych warunkach. Dlatego używane w astrofotografii sensory
światła muszą być bardzo czułe, a czasy naświetlania zdjęć dużo większe niż
w klasycznej fotografii.

Kolejnym problemem są niewielkie rozmiary kątowe fotografowanych obiektów. Tu
też są wyjątki, przykładowo droga mleczna, która zajmuje praktycznie całe nasze
niebo. Jest jednak wiele interesujących obiektów, których rozmiary kątowe
liczymy w minutach kątowych\footnote{Minuta kątowa - 1/60 stopnia}. Do robienia
takich zdjęć nie wystarczy zwykły obiektyw, potrzebne są większe i bardziej
zaawansowane układy optyczne o większych ogniskowych.

Ważny jest też fakt, że niebo, które fotografujemy, porusza się stale względem
nas. Ten pozorny dobowy ruch wszystkich ciał niebieskich wynika z ruchu
obrotowego ziemi, podobnie jak ruch słońca. Prędkość tego ruchu jest stała
i wynosi dokładnie jeden obrót na dobę gwiazdową\footnote{Doba gwiazdowa - okres
obrotu ziemi wokół własnej osi, wynoszący ok 86164s}.  O ile w klasycznej
fotografii taka prędkość nie stanowi problemu, to w fotografii nieba, biorąc pod
uwagę potrzebę długich czasów naświetlania, jest to istotny problem. Zrobienie
udanego zdjęcia wymaga, aby obiekt nie poruszał się przez czas naświetlania.
Biorąc dodatkowo pod uwagę długie ogniskowe teleskopu problem efektywnego
śledzenia ruchu ciał niebieskich okazuje się kluczowy w astrofotografii.

\section{Kamery używane w astrofotografii}

Zdjęcia o długich czasach ekspozycji stanowią duże wyzwanie dla kamer. Jednym
z głównych problemów jest szum zakłócający wynikowy obraz, przede wszystkim szum
termiczny. Dlatego używane w kamerach cyfrowych matryce muszą być szczególnie
dobrej jakości, a w bardziej profesjonalnych zastosowaniach są one chłodzone do
bardzo niskich temperatur (poniżej $-30\degree C$), aby zminimalizować szum
termiczny.

W astrofotografiii amatorskiej popularnym i ekonomicznym rozwiązaniem jest
stosowanie lustrzanek cyfrowych. Są one przystosowane do robienia zdjęć
o dowolnych czasach ekspozycji, a do tego łatwo dostępne i nie wymagające
żadnych modyfikacji. Wśród lustrzanej cyfrowych najbardziej popularne są
kolorowe matryce CMOS oraz CCD.

Dedykowane kamery astronomiczne pozwalają na dużo lepszą jakość zdjęć.
Największą popularność mają kamery korzystające z monochromatycznych matryc CCD.
Te urządzenia są jednak znacznie droższe od lustrzanek cyfrowych.

Do fotografii planet i księżyca często stosuje się zwykłe kamery internetowe,
ponieważ nie są w tym przypadku potrzebne duże czasy naświetlania, ale jak
najwięcej ekspozycji o krótkich czasach.

\section{Teleskopy}

W obserwacji nieba w zakresie światła widzialnego stosuje się dwa podstawowe
rodzaje układów optycznych: zwierciadlane (korzystające ze zjawiska odbicia
światła) i soczewkowe (korzystające ze zjawiska załamania, zwane też lunetami).
Istnieją także układy zwierające zarówno soczewki jak i zwierciadła (np.
teleskop Schmidta czy teleskop Maksutowa). 

Układy optyczne są zawsze obarczone wadami optycznymi:

\begin{itemize}

	\item Aberracja sferyczna to wada występująca w zwierciadłach
		i soczewkach sferycznych. Promienie przechodzące przez układ
		optyczny w różnej odległości od osi optycznej nie ogniskują się
		w jednym miejscu.
	
	\item Aberracja chromatyczna, występująca w lunetach, wynika z różnego
		kąta załamania światła o różnych długościach fali. Każda
		soczewka działa jak pryzmat, powodując rozszczepienie światła na
		składowe.

	\item Koma to wada polegająca na rozciągnięciu obrazu punktów nie
		leżących na osi optycznej. W efekcie gwiazdy leżące na granicy
		pola widzenia teleskopu przyjmują kształt przecinka. Koma
		występuje w zwierciadłach parabolicznych.

\end{itemize}

\section{Współrzędne sferyczne}

\begin{figure}
\label{fig_sfera}
\includesvg{sfera.svg}
\caption{Współrzędne na sferze niebieskiej}
\end{figure}

Rozpatrując pozycje i ruchy ciał niebieskich będziemy poruszali się na
\emph{sferze niebieskiej}. Sfera niebieska to pojęcie abstrakcyjne oznaczające
sferę o dowolnym promieniu i o środku w miejscu obserwacji.  Półprosta
poprowadzona z miejsca obserwacji, przecinająca dany obiekt, przecina sferę
niebieską w punkcie zwanym \emph{położeniem} tego obiektu na sferze niebieskiej
\cite{rybka}.

Promień sfery niebieskiej nie ma dla nas znaczenia jako dla obserwatorów.
Interesuje nas jedynie położenie na sferze. Dobrą analogią do tej sytuacji jest
określanie położenia na kuli ziemskiej za pomocą współrzędnych geograficznych:
długości i szerokości. Podobnie będziemy określali współrzędne na sferze
niebieskiej - za pomocą analogicznych dwóch miar kątów. Jedynym co musimy
ustalić jest położenie biegunów i środek układu współrzędnych.

W tej pracy korzystać będziemy głównie z układu \emph{równikowego}. Bieguny
znajdują się w nim w punktach przecięcia osi obrotu ziemi ze sferą niebieską.
Daje to tę podstawową zaletę, że ciała niebieskie mają niezmienną odległość od
tych punktów, więc w ciągu ich dobowego ruchu po sferze niebieskiej jedna ze
współrzędnych (odpowiadająca szerokości geograficznej) będzie niezmienna. Ta
współrzędna to \emph{deklinacja} oznaczana grecką literą $\delta$.

Podobnie jak w układzie współrzędnych geograficznych, na sferze niebieskiej
możemy wyznaczyć okrąg wielki będący \emph{równikiem niebieskim}, czyli zawarty
w płaszczyźnie prostopadłej do osi obrotu sfery niebieskiej. Punkty na równiku
mają zerową deklinację.

Ustalenie drugiej współrzędnej wymaga wybrania południka zerowego na niebie
(odpowiednika południka zerowego w układzie współrzędnych geograficznych).
Istnieją dwa rodzaje układu współrzędnych tego typu, tj. układu współrzędnych
równikowych.  Pierwszy z nich, układ \emph{równikowy godzinny}, za południk
zerowy obiera \emph{południk lokalny}, czyli południk przechodzący przez zenit.
Odpowiednikiem długości geograficznej jest w nim \emph{kąt godzinny}. W ciągu
dobowego ruchu sfery niebieskiej południk lokalny jest niezmienny względem
obserwatora, ale zmieniają się kąty godzinne ciał niebieskich. W drugim układzie
współrzędnych, \emph{równikowym równonocnym}, południk zerowy przebiega przez
punkt równonocy wiosennej, inaczej punkt barana oznaczany symbolem $\aries$.
Odpowiednikiem długości geograficznej jest w nim \emph{rektascensja} oznaczana
grecką literą $\alpha$ która jest stała i niezależna od miejsca obserwacji.
Układ równikowy równonocny jest najczęściej stosowany w atlasach nieba do
określania bezwzględnej pozycji ciał niebieskich.

Deklinację mierzymy zazwyczaj w stopniach i zawiera się w przedziale od
$-90\degree$ do $90\degree$. Rektascensję również możemy mierzyć w stopniach,
jednak zazwyczaj podaje się ją w mierze czasowej, od $0$ do $24h$.

\section{Montaż teleskopu}

W astronomii obserwacyjnej instrumenty obserwacyjne korzystają z dwóch typów
mocowania teleskopu (montaży): azymutalnego i paralaktycznego. Montaż azymutalny
jest prostszy w konstrukcji i w użyciu, korzysta bowiem z naturalnego układu
współrzędnych horyzontalnych. Takie montaże są jednak mało praktyczne
w astrofotografii amatorskiej. Montaż paralaktyczny jest natomiast bardzo
użyteczny w fotografowaniu nieba, ponieważ porusza się w opisanym wyżej układzie
równikowym. Oznacza to, że do śledzenia ruchu fotografowanych obiektów wystarczy
zmiana w tylko jednej osi. 

Takie montaże to zazwyczaj bardzo precyzyjne urządzenia. Załóżmy parametry
typowe dla amatorskiego zestawu do astrofotografii: ogniskowa teleskopu 750mm
i matryca lustrzanki cyfrowej o rozmiarach 22.2mm x 14.8mm. Zdjęcie będzie
obejmowało obszar na niebie o kątowych rozmiarach:

$$2*arc tg(d/2f)$$

Czyli dla powyższych parametrów ok. 1,7 x 1,3 stopnia. Zakładając rozdzielczość
finalnego zdjęcia rzędu 1000px, każdy piksel ma rozmiar rzędu kilku sekund
kątowych. Każde przesunięcie o 10'' jest widoczne na takim zdjęciu.

Aby uzyskać żądaną dokładność montaże astronomiczne zawierają wbudowane
przekładnie umożliwiające bardzo niewielkie przesunięcia, rzędu sekund
kątowych lub mniejsze. Ręczna korekcja pozycji na bieżąco jest oczywiście
niedokładna i wyjątkowo męcząca, biorąc pod uwagę czasy ekspozycji zdjęć.
Dlatego montaże wyposaża się w mechanizmy automatycznie obracające teleskopem.
W najprostszym scenariuszu wystarcza obrót w jednej osi (osi kąta godzinnego) ze
stałą prędkością. W efekcie obserwowany obiekt pozostaje w polu widzenia
teleskopu bez potrzeby ręcznej korekcji.

\section{Systemy GoTo}

Jednym z kolejnych praktycznych problemów przy wykonywaniu zdjęć niewielkich
obiektów głębokiego nieba jest ich wyszukiwanie na niebie. Można robić to
ręcznie naprowadzając oś optyczną teleskopu na wybrany punkt na niebie np.
z pomocą niewielkiej lunety dołączanej do głównego teleskopu (tzw. szukacz).
Kiedy jednak do głównego teleskopu dołączona jest kamera, trudno jest użyć go do
potwierdzenia, że ustawiliśmy teleskop poprawnie. Ciężko jest tu zastosować
techniki znane z obserwacji wizualnych.  Ręczne ustawianie teleskopu wymaga
w takiej sytuacji wiele doświadczenia i dobrej znajomości nieba, a zajmuje dużo
cennego czasu, który moglibyśmy poświęcić na ekspozycję zdjęcia.

Rozwiązaniem tego problemu jest system GoTo. Korzysta on z napędu w obu osiach
montażu do ustawiania go w żądanej pozycji. Po szybkiej kalibracji na jasnych
obiektach (np. gwiazdach) system jest gotowy do użycia.

\section{Zaawansowane metody śledzenia ruchu sfery niebieskiej}

Ostatnim usprawnieniem, o jakim będę wspomniał w tej pracy jest tzw.
\emph{guiding}.  Rozwiązuje on problem dokładnego śledzenia obiektów, jednak
w bardziej wyrafinowany sposób niż sterowanie napędem w jednej osi. Teoretycznie
powinien on działać wystarczająco dobrze, na drodze stają nam jednak dwa
problemy: niedokładne wykonanie przekładni i niedokładne ustawienie montażu.
Pierwszy problem powoduje efekt niejednostajnego ruchu montażu pomimo
jednostajnego obrotu śruby mikroruchów. Drugi problem zależy od wprawy
użytkownika montażu, który przed użyciem musi ustawić jego oś dokładnie tak, aby
przebiegała przez biegun niebieski. W praktyce to ustawienie nigdy nie jest
idealne i po odpowiednio długim czasie śledzenia następuje widoczne przesunięcie
w osi deklinacji.

Systemy guidingu polegają na zastosowaniu drugiego, mniejszego teleskopu
z czułą kamerą, przymocowanego do głównego teleskopu, do rejestrowania obrazu
nieba w czasie rzeczywistym. Ten obraz jest na bieżąco analizowany pod kątem
niepożądanych przesunięć, które są później korygowane przez odpowiedni moduł
elektroniczny.

W praktyce za analizę obrazu i korekcję ruchu odpowiada zazwyczaj zewnętrzny
komputer, sterownik teleskopu ma jedynie za zadanie umożliwić wprowadzenie
takich korekcji z zewnątrz.
