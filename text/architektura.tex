\section{Przypadki użycia}

Zgodnie z opisaną wcześniej problematyką można wyróżnić kilka przypadków
użycia projektowanego systemu:

Śledzenie: użytkownik włącza tryb śledzenia w urządzeniu. Efektem powinno być
automatyczne śledzenie ruchu nieba.

Ręczna zmiana pozycji: użytkownik korzysta z przycisków pilota do ręcznej
korekcji pozycji montażu.

Kalibracja współrzędnych: użytkownik ręcznie ustawia montaż na znany sobie
obiekt (np. jasną gwiazdę), po czym zapisuje do urządzenia jego pozycję
w układzie współrzędnych równikowych równonocnych. Sterownik zapamiętuje
pozycję, w której się znajduje.

Ustawienie montażu w wybranej pozycji (GoTo): użytkownik ustawia pozycję
docelową montażu i ustawia sterownik w tryb GoTo. Sterownik rozpoczyna rotację
montażu w obu osiach w odpowiednich kierunkach.

Wyłączenie kontroli nad montażem: użytkownik wyłącza kontrolę silników nad
montażem.

\section{Tryby pracy}

Sterownik montażu może znajdować się w kilku stanach:

\begin{itemize}

\item Śledzenie - tryb, w którym pozycja montażu zmienia się wraz z ruchem
nieba.  Sterowniki silników skonfigurowane są w trybie pracy mikrokrokowej.

\item GoTo - tryb dążenia do pozycji docelowej. Silniki pracują w trybie pełnych
kroków. Po ustawieniu w odpowiedniej pozycji system przechodzi w tryb śledzenia.

\item Tryb ręczny - kierunek pracy silnika ustawiany jest ręcznie przez
użytkownika.

\item Wyłączony - silniki są wyłączone.

\end{itemize}

Z punktu widzenia użytkownika dwa z tych stanów, GoTo i tryb ręczny, wyróżniają
się tymczasowością. Sterownik nie powinien pozostawać w tych stanach długo,
a użytkownik nadzoruje wtedy pracę sterownika. Stany wyłączony i śledzenie są
natomiast przeznaczone do nieprzerwanej pracy przez dłuższy okres czasu (rzędu
kilku godzin). Można więc przyjąć, że pierwsze dwa tryby pracy są \emph{ręczne},
natomiast pozostałe dwa są \emph{autonomiczne}.

Istotne jest, żeby domyślnym trybem pracy urządzenia był jeden z trybów
autonomicznych, ponieważ ręczne tryby pracy bez nadzoru mogą narazić na
uszkodzenie sprzęt obserwacyjny i nie powinny być używane bez nadzoru.

Typową sytuacją jest pozostawienie ustawionego montażu na dłuższy czas w celu
zebrania pewnej ilości danych (np. ekspozycja zdjęcia). Jeśli w takim
scenariuszu tryb śledzenia zostanie z jakiegoś powodu przerwany, dalsze
zbieranie prawidłowych danych będzie niemożliwe do czasu reakcji użytkownika.
Dlatego w przypadku awarii takich jak chwilowa utrata zasilania, powodujących
ponowną inicjalizację systemu, urządzenie wejdzie w jeden z trybów
autonomicznych, który pozwoli na kontynuację samodzielnej pracy.

Zależnie od trybu pracy urządzenia inne będą ustawienia pracy silników. Tabela
\ref{tab_settings} przedstawia ustawienia dla poszczególnych trybów.

\begin{table}[t]

\caption{Ustawienia sterownika dla różnych trybów pracy} \label{tab_settings}

\begin{tabular}{|l|l|l|l|l|}

\hline Tryb	& Długość kroku	& Rozdzielczość mikrokroków \\

\hline Śledzenie & 0.207s	& 1/16 \\

\hline GoTo	& 0.002s	& 1 \\

\hline Ręczny	& 0.02s		& 1/16 \\

\hline Wyłączony & -		& - \\

\hline 

\end{tabular}

\end{table}

\section{Struktura oprogramowania}

Oprogramowanie projektu dzieli się na dwie zasadnicze części: sterownik montażu
i aplikację klienta na urządzenia mobilne.

\subsection{Sterownik montażu}

Implementacja sterownika korzysta z bibliotek SDK dostarczonego przez Nordic
Semiconductors, które tworzą podstawową warstwę abstrakcji sprzętu.
