\section{Przegląd użytych komponentów sprzętowych}

\subsection{nrf51822}

Głównym modułem zaprojektowanego urządzenia jest nrf51822. Realizuje on
jednocześnie komunikację (Bluetooth) oraz logikę sterownika.

Moduł nrf51822 oparty jest o 32-bitowy układ ARM Cortex M0. Umożliwia
komunikację z użyciem Bluetooth Low Energy.

\subsection{A4988}

Komponentem odpowiedzialnym za sterowanie silnikami jest moduł A4988. Jest to
sterownik silników krokowych z możliwością pracy w trybie mikrokroków.

Napięcie zasilania silników: 8V - 35V

Maksymalne natężenie prądu: 2A

Możliwość pracy w trybach pełnego kroku, 1/2 kroku, 1/4 kroku, 1/8 kroku
i 1/16 kroku

Moduł posiada układ ograniczający prąd pobierany przez silnik, korzystający
z modulacji PWM. Limit prądu można ustawić za pomocą potencjometru wbudowanego
w układ.  Maksymalna wartość prądu wynosi:

$$I_{TripMax} = V_{ref}/(8*R_s)$$

Gdzie $R_s$ jest rezystancją czujnika wynoszącą dla użytego układu $0.05\ohm$
a $V_{ref}$ napięciem na pinie REF (możliwym do łatwego zmierzenia).

Ograniczenie prądu umożliwia również pracę z silnikami o niższym napięciu
nominalnym niż 8V.

\subsection{Silniki}

Urządzenie pracuje z bipolarnymi silnikami krokowymi o nominalnych wartościach
prądu i napięcia mniejszych niż limit sterownika: 2A i 35V. Przed użyciem
silników należy się upewnić co do limitu prądu ustawionego w sterowniku:
powinien być on równy wartości nominalnej prądu dla użytego silnika.

Silniki użyte do testów urządzenia to bipolarne silniki krokowe o nominalnych
wartościach napięcia i prądu odpowiednio 12V i 400mA.

Liczba kroków na pełen obrót użytego silnika krokowego jest konfigurowana w kodzie sterownika
jako MOTOR\_STEP\_RATIO i dla testowanych silników wynosi 200. Użycie innej
wartości (np 400 kroków dla niektórych silników) wymaga ponownej kompilacji
kodu sterownika.
